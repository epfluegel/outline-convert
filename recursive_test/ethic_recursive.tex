\documentclass{beamer}
\usepackage[T1]{fontenc}
\usepackage{graphicx}
\usetheme{Goettingen}
\definecolor{links}{HTML}{2A1B81}
\hypersetup{colorlinks,linkcolor=,urlcolor=links}
\title{Ethical and Legal Aspects of Security}
\date{\today}
\AtBeginSection[]
{
  \begin{frame}<beamer>{Outline}
      \tableofcontents[currentsection, currentsubsection]
  \end{frame}
}
\begin{document}
\begin{frame}
  \titlepage
\end{frame}

\begin{frame}{Learning Objectives -- Legal and Ethical Issues}
\begin{itemize}
  \item Know the definition of cyber crime and discuss the main challenges around this type of crime
  \item Understand legal frameworks for protecting against the theft of computer assets
  \item Know typical threats to privacy, and outline suitable controls to these
  \item Discuss ethical issues in computer/cyber security
  \begin{itemize}
    \item Know the definition of a code of ethics
    \item Discuss the scenario and dilemmas involved in a range of scenarios
  \end{itemize}
\end{itemize}
\end{frame}
\begin{frame}{Motivation}
\begin{itemize}
  \item The term "ethical hacker" suggests already: there are many ethical and legal issues in the field of computer security.
  \item In your opinion, what might be the most critical issues?
  \item We will discuss:
  \begin{itemize}
    \item Computer/Cybercrime
    \item Legal Aspects of Computer Security
    \item Ethical Issues in Computer Security
  \end{itemize}
\end{itemize}
\end{frame}
\section{Legal Aspects in Computer Security}
\subsection{Computer and Cybercrime}
\begin{frame}{Computer Crime is a Separate Category}
\begin{itemize}
      \item Traditional legislation sometimes difficult to apply
      \begin{itemize}
        \item Legislation is out of pace -- creating and changing laws are slow processes, people involved don't understand fast progressing technology (computers, Internet, Web technologies..)
        \item Difficult to define the ownership of data/software -- in some cases: courts failed to acknowledge theft
      \end{itemize}
\end{itemize}
\end{frame}
\begin{frame}{Challenges for Fighting Computer Crime}
\begin{itemize}
      \item Difficult to provide evidence
      \begin{itemize}
        \item Lack of physical evidence -- there are no fingerprints!
        \item Example: how to prove that system has been accessed by a particular individual?
        \item Digital Forensics is emerging new field
      \end{itemize}
      \item Hard to prosecute
      \begin{itemize}
        \item Computer crimes could be committed by juveniles
        \item International dimensions due to networked communications
      \end{itemize}
      \item Complexity of case -- difficult to understand what exactly happened
\end{itemize}
\end{frame}
\begin{frame}{Definition of Computer Crime}
\begin{itemize}
\item \href{https://en.wikipedia.org/wiki/Computer_Misuse_Act_1990}{Computer Misuse Act} (UK, 1990) -- controversial and ahead of its time
      \item [Computer Misuse Act](https://en.wikipedia.org/wiki/Computer\_Misuse\_Act\_1990) (UK, 1990) -- controversial and ahead of its time
      \item May 2007: European Commission noted that there was not even an agreed definition of cybercrime
      \item It proposed a threefold definition:
      \begin{itemize}
        \item Traditional forms of crime such as fraud or forgery, though committed over electronic communication networks and information systems;
        \item The publication of illegal content over electronic media, e.g.
        \begin{itemize}
          \item child abuse material
          \item incitement to racial hatred
        \end{itemize}
        \item Crimes unique to electronic networks, e.g., attacks against information systems, denial of service and hacking.
      \end{itemize}
\end{itemize}
\end{frame}
\begin{frame}{Mini-Exercise}
\begin{itemize}
      \item Briefly researching the Internet, can you find a recent story on cyber crime in the news?
      \item Which category/type of computer/cyber crime was involved?
\end{itemize}
\end{frame}
\subsection{Legal Computer Security Controls}
\begin{frame}{[COMPLETE] Threats to Critical Assets}
\begin{itemize}
      \item Let us investigate where the law can be used as a computer security control
      \item We will first study threats that might involve assets being stolen or leaked to competitors:
      \begin{itemize}
        \item Hardware
        \begin{itemize}
          \item Computer hardware design is an important IP
        \end{itemize}
        \item Software
        \begin{itemize}
          \item Source code is vital for making profit from an application
          \item Object/binary code needs to be protected, too
        \end{itemize}
        \item Data/Information
        \begin{itemize}
          \item Traditionally, sold items were "things" or "services"
          \item Information doesn’t fit easily into these categories (is not a thing but can be replicated)
          \item Difficult to use traditional legislation (see previous topic)
        \end{itemize}
      \end{itemize}
\end{itemize}
\end{frame}
\begin{frame}{[COMPLETE] Threats to Critical Assets (cont.)}
\begin{itemize}
      \item Another asset is the People category
      \item Main threat considered here is violation of privacy
      \begin{itemize}
        \item This is confidentiality of sensitive data about user (personal information)
        \item Might be exploited by businesses, government or malicious individuals
      \end{itemize}
      \item Used to be only magazine subscriptions, loyalty cards, credit card transactions, phone calls, CCTV recordings
      \item Threats to Privacy are now on completely new scale
      \begin{itemize}
        \item Cloud computing (Dropbox, Box, OneDrive, Google Drive, iCloud)
        \item Social networking (Facebook, TikTok, Instagram, LinkedIn, X formerly known as Twitter, BlueSky)
        \item Instant messaging (WhatsApp, Telegram, Signal, Viber, Threema)
        \item AI Interfaces (ChatGPT, Gemini, Copilot, DeepSeek)
      \end{itemize}
\end{itemize}
\end{frame}
\begin{frame}{[COMPLETE] Applying Traditional Legal Protection to Computer Assets}
\begin{itemize}
      \item Legal security controls establish enforceable obligations, restrictions, and expectations that protect digital assets, proprietary information, and service reliability.
      \item Overview of legal controls:
      \begin{itemize}
        \item Contracts
        \begin{itemize}
          \item Legal protections include non-disclosure agreements (NDAs)
          \item Service Level Agreement (SLA)
        \end{itemize}
        \item Trade Secrets
        \item Trademarks
        \item Copyrights
        \item Software Licensing
        \item Patents
      \end{itemize}
      \item They have various degrees of effectiveness
\end{itemize}
\end{frame}
\begin{frame}{[COMPLETE] Trade Secrets}
\begin{itemize}
      \item A trade secret is confidential information that is of economic benefit on its owner
      \item E.g. business information that provides a competitive advantage, such as proprietary algorithms, formulas (like the Coca-Cola recipe), or customer lists.
      \item Protection relies on keeping the information secret through legal agreements (e.g., NDAs) and other security measures.
      \item Violation of the agreement generally carries the possibility of heavy financial penalties
      \item Can protect:
      \begin{itemize}
        \item Data
        \item Software
      \end{itemize}
\end{itemize}
\end{frame}
\begin{frame}{[COMPLETE] Trademarks}
\begin{itemize}
      \item A recognisable symbol, word, phrase, or logo that distinguishes a brand’s products or services from others (e.g., the Nike swoosh or the word "Google").
      \item Trademarks are legally protected through registration with government agencies
      \item They give the owner exclusive rights to use the mark in commerce.
      \item Protection lasts as long as the trademark is actively used and renewed.
\end{itemize}
\end{frame}
\begin{frame}{[COMPLETE] Trade Secrets and Trademarks}
\begin{itemize}
      \item The main difference between a trade secret and a trademark lies in their purpose and how they are protected:
      \item In short, trade secrets protect secret information, while trademarks protect brand identity in the marketplace.
\end{itemize}
\end{frame}
\begin{frame}{[COMPLETE] Mini-Exercise -- Which Legal Control Example?}
\begin{itemize}
      \item A cybersecurity software company would like to protect its logo and brand name to prevent others from creating counterfeit or malicious software that could be mistaken for their legitimate product, protecting users from security risks.
      \item A company hires a third-party IT firm to manage its cybersecurity and includes clauses that require the firm to implement encryption, access controls, and regular security audits. If the firm fails to meet these terms, the company has legal grounds to seek damages.
      \item A cloud service provider agrees to maintain 99.9\% uptime and implement strong security measures, such as data encryption and incident response protocols. If they fail to meet these commitments, customers may be entitled to compensation or contract termination.
\end{itemize}
\end{frame}
\begin{frame}{[COMPLETE] Copyrights}
\begin{itemize}
      \item Different countries have different copyright concepts and legislations.
      \item Traditional purpose:
      \begin{itemize}
        \item Applies to creative work
        \item Example: music, art, writing
      \end{itemize}
      \item Computer assets:
      \begin{itemize}
        \item Data
        \item Source code of program
      \end{itemize}
      \item Can be enforced through technology (Digital Rights Management).
\end{itemize}
\end{frame}
\begin{frame}{[COMPLETE] Software Licensing}
\begin{itemize}
      \item A license is a form of copyright protection
      \item Proprietary software uses this to protect:
      \begin{itemize}
        \item Source code
        \item Object code
      \end{itemize}
      \item Problems:
      \begin{itemize}
        \item legal enforcement
        \item hackers may obtain source code anyway through reverse engineering
      \end{itemize}
\end{itemize}
\end{frame}
\begin{frame}{[COMPLETE] Open Source Licenses}
\begin{itemize}
      \item Open Source software is released with source code
      \item Distributed via servers such as GitHub
      \item Very successful model, with many advantages
      \item For OS, alternative license models exist:
      \begin{itemize}
        \item GPL
        \begin{itemize}
          \item "Copyleft" -- Notion invented by Stallman (Free Software Foundation)
          \item Anyone may copy/modify, provided source code remains free
        \end{itemize}
        \item Creative Commons
        \item Many others..
      \end{itemize}
\end{itemize}
\end{frame}
\begin{frame}{[COMPLETE] Patents}
\begin{itemize}
      \item Traditional purpose:
      \begin{itemize}
        \item Protects device or process for carrying out an idea
        \item Not the idea itself
      \end{itemize}
      \item Computer assets:
      \begin{itemize}
        \item Hardware can be patented
        \item For a long time, algorithms were seen as fact of nature
        \item In some cases now, computer programs can be subject to patents
        \item Example: RSA Algorithm -- US patent expired in 2000
      \end{itemize}
\end{itemize}
\end{frame}
\begin{frame}{[COMPLETE] Controls Protecting Privacy}
\begin{itemize}
      \item Legal
      \begin{itemize}
        \item General Data Protection Regulation (EU/UK) (GDPR)
        \item Data Protection Act (UK) 2018
      \end{itemize}
      \item Behavioural
      \begin{itemize}
        \item Be aware of social engineering threats
        \item Moderate your active content on social media
        \item Always shred your bills and receipts
      \end{itemize}
      \item Technical
      \begin{itemize}
        \item Achieve Anonymity (Tor, VPNs, Signal)
      \end{itemize}
\end{itemize}
\end{frame}
\begin{frame}{[COMPLETE] Mini-Exercise: Open Source Code}
\begin{itemize}
      \item Briefly researching the Internet, can you find any other open source licensing schemes?
      \item What are the main differences?
      \item If you are a software developer, writing a Google Chrome extension, which type of licence will you use?
      \item Debate: does the Open Source model make software more secure?
\end{itemize}
\end{frame}
\section{Ethical Issues in Computer Security}
\begin{frame}{Definition of Ethical System}
\begin{itemize}
    \item Ethical System
    \begin{itemize}
      \item Tells us what might be acceptable behaviour in a society
      \item Is different to a law
      \item Is not simply right or wrong
      \item Might differ between different groups/individuals
      \item Ethical positions often come into conflict ("dilemmas")
    \end{itemize}
    \item We can identify some example dilemma scenarios:
    \begin{itemize}
      \item Is grey hat hacking ethical?
      \item Should information on security vulnerabilities be published?
      \item Should organisations or individuals be allowed to create encryption that the government cannot break?
    \end{itemize}
\end{itemize}
\end{frame}
\begin{frame}{Vulnerability Disclosure Dilemma}
\begin{itemize}
    \item Suppose an individual has found a security vulnerability in an application.
    \item What should he/she do, and what might be ethical implications?
    \item What are the dilemmas?
\end{itemize}
\end{frame}
\begin{frame}{Discussion of Dilemma}
\begin{itemize}
    \item Publish the information openly:
    \begin{itemize}
      \item This forces the writers of the application to provide patches
      \item It is important that users are alerted and can protect themselves
      \item It helps to keep security in the headlines
      \item Dilemma: this might encourage (zero-day) security attacks
    \end{itemize}
    \item Contact the authors/vendors of the software directly
    \begin{itemize}
      \item Bounties are available!
      \item Dilemma: company might not listen.
    \end{itemize}
    \item Recent trend:
    \begin{itemize}
      \item Hackers sell information to criminals who carry out zero-day attacks
      \item Or they assist government agencies (FBI)
    \end{itemize}
\end{itemize}
\end{frame}
\begin{frame}{Grey-Hat Hacking}
\begin{itemize}
    \item 
    \item Definition of Gray Hat Hacker
    \begin{itemize}
      \item White Hat (Ethical) Hackers -- non-malicious attackers employed to find vulnerabilities in an organisation to improve security.
      \item Grey Hat Hackers -- somewhere between White Hat and Black Hat. For example, they might carry out ethical hacking activities without consent.
      \item Black Hat Hackers (Computer Criminals) -- malicious attackers who want to profit financially.
    \end{itemize}
\end{itemize}
\end{frame}
\begin{frame}{The Ethics of Grey-Hat Hacking}
\begin{itemize}
    \item Controversy on the act of unsolicited ethical computer hacking
    \item What is your opinion:
    \begin{itemize}
      \item Does this raise ethical issues?
      \item Should we condemn this "profession"?
      \item What are the dilemmas?
    \end{itemize}
    \item To better understand this, we need to study possible motivations behind attacks.
\end{itemize}
\end{frame}
\begin{frame}{Strategic Security Games}
\end{frame}
\begin{frame}{Interacting with a Grey-Hat Hacker}
\end{frame}
\begin{frame}{The Bimatrix Grey-Hat Hacker Game}
\end{frame}
\begin{frame}{Grey-Hat Hacker Game: Strategic Normal Form}
\end{frame}
\begin{frame}{Dilemma Discussion}
\begin{itemize}
    \item Some Grey Hat attackers claim:
    \begin{itemize}
      \item We don’t intend to cause harm
      \item We help making systems more secure
      \item We do research
    \end{itemize}
    \item On the other hand:
    \begin{itemize}
      \item Can we really trust them?
      \item There are grey hat hackers turning to malicious hackers/computer criminals
      \item A lot of these activities might never be detected and might not improve security
      \item Is an unlawful action after all
    \end{itemize}
\end{itemize}
\end{frame}
\end{document}