\documentclass{beamer}
\usepackage[T1]{fontenc}
%\setbeamertemplate{frametitle continuation}{}
\usetheme{Goettingen}
\definecolor{links}{HTML}{2A1B81}
\hypersetup{colorlinks,linkcolor=,urlcolor=links}
\title{CI7526 DAY 1}
\subtitle{}
\author{author\inst{1}}
\institute[Universities of Somewhere and Elsewhere]{
  \inst{1}%
  School of Computer Science and Mathematics
  Kingston University
}
\date{\today}
\AtBeginSection[]
{
  \begin{frame} <beamer> {Outline}
    \tableofcontents[currentsection, currentsubsection]
  \end{frame}
}
\begin{document}
\begin{frame}
  \titlepage
\end{frame}
\begin{frame}[allowframebreaks]{CI7526 DAY 1}
\begin{itemize}
  \item LECTURE \#wfe-ignore-item
  \begin{itemize}
    \item BASIC SECURITY + AI STORY \#wfe-ignore-item
    \begin{itemize}
      \item BASIC \#linear \#basic \#wfe-ignore-item
      \begin{itemize}
        \item Assets \#CI7100 \#CI7130\_Intro \#slide \#CI7160-intro \#basic
        \begin{itemize}
          \item In order to better understand security attacks, we have to define what we want to protect
          \item This is called an asset
          \item We categorise assets in the following way:
        \end{itemize}
        \item Assets \#slide \#CI7100 \#CI7130\_Intro \#CI6240\_SM-1 \#basic
        \begin{itemize}
          \item ![assets.png](https://www.dropbox.com/s/wy569h036gsfa0v/assets.png?dl=1) \#wfe-style:normal
        \end{itemize}
        \item Mini-Exercise: Assets \#slide \#CI7100 \#CI7130\_Intro \#basic
        \begin{itemize}
          \item Look around you. There are many assets of the Kingston University system present. List some asset items, with at least one in each category.
          \item Which of them do you consider being critical assets -- these would be assets of particular value or importance for your student experience.
        \end{itemize}
        \item The Goals of Computer Security \#slide \#CI7100 \#CI7160-intro \#CI7130\_Intro \#basic
        \begin{itemize}
          \item We will define as the main goals (or requirements) of Computer Security (“CIA”):
          \item This terminology is adopted by academia and industry
          \item Note: cannot “guarantee” absolute security
        \end{itemize}
        \item Confidentiality \#slide \#CI7100 \#CI7160-intro \#CI7130\_Intro \#basic
        \begin{itemize}
          \item Definition: computer related assets must be accessed only by authorised parties
          \item Synonyms: secrecy, privacy
          \item Example: you need to prevent hackers from accessing personal data in your social network account
          \item There may be different levels of confidentiality for different users
        \end{itemize}
        \item Integrity \#slide \#CI7100 \#CI7160-intro \#CI7130\_Intro \#basic
        \begin{itemize}
          \item Data Integrity:
          \item Origin Integrity (Authentication):
          \item Non-repudiation:
        \end{itemize}
        \item Availability \#slide \#CI7100 \#CI7160-intro \#CI7130\_Intro \#basic
        \begin{itemize}
          \item Definition: assets are accessible to authorised parties at appropriate times
          \item In other words: legitimate access should always be possible
          \item Example: a website should be available 24/7.
          \item Different availability levels
        \end{itemize}
        \item Confidentiality, Integrity and Availability \#slide \#CI7100 \#CI7160-intro \#CI7130\_Intro \#basic
        \begin{itemize}
          \item ![CIA.png](https://www.dropbox.com/s/i12op0g3xmcby27/CIA.png?dl=1) \#eyo-style:Normal
        \end{itemize}
        \item Threats, Vulnerabilities, Attacks and Risks \#slide \#CI7100 \#CI7160-intro \#CI7130\_Intro \#basic
        \begin{itemize}
          \item Threat: set of circumstances with potential to exploit vulnerability and create a risk of harm
          \item Vulnerability: potentially exploitable weakness of an asset
          \item Attack: realisation of a threat ("threat event")
          \item Risk: occurs if “matching” threat and vulnerability exist
        \end{itemize}
        \item Controls \#slide \#CI7100 \#CI7160-intro \#CI7130\_Intro \#basic
        \begin{itemize}
          \item A control is used for protection.
          \item It could be for example an
          \item It either eliminates the threat and/or closes the vulnerability.
          \item Controls need to be effective.
        \end{itemize}
        \item \#todo Basic Security Methodology \#slide \#CI6240\_SM-1 \#CI7100 \#CI7160\_D1 \#CI7130\_Intro \#basic
        \begin{itemize}
          \item The basic methodology in computer security is:
          \item Impossible to guarantee security, can only minimise risk
          \item Need for periodic review
        \end{itemize}
        \item Generic Security Methodology \#slide \#CI7160-intro
        \begin{itemize}
          \item For each critical asset do:
          \item Constant re-evaluation is necessary
        \end{itemize}
        \item Mini-Exercise: Security Goals \#slide:mini \#CI7100 \#CI7130\_Intro \#basic
        \begin{itemize}
          \item Read the story about the recent security incident at [The Register](https://www.theregister.co.uk/security/).
          \item Analyse the story more in-depth (you may have to research older, related articles) and identify which of the main security goals were violated, and of which (critical) assets.
        \end{itemize}
      \end{itemize}
    \end{itemize}
  \end{itemize}
  \item WORKSHOP
  \begin{itemize}
    \item WRITE UP STORY AS CASE STUDY
    \item CHATGPT
    \item CWK
  \end{itemize}
\end{itemize}
\end{frame}
\end{document}